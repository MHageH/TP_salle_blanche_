\documentclass[11pt]{article}
\renewcommand{\baselinestretch}{1.05}

\usepackage{amsmath,amsthm,verbatim,amssymb,amsfonts,amscd, graphicx}
\usepackage{graphics}

\usepackage{xcolor}

\usepackage[hidelinks]{hyperref}

\usepackage{parskip}

\renewcommand{\contentsname}{Table des mati\`eres}

\topmargin0.0cm
\headheight0.0cm
\headsep0.0cm
\oddsidemargin0.0cm
\textheight23.0cm
\textwidth16.5cm
\footskip1.0cm

\begin{document}

\title{\textbf{Technologie salle blanche et caract\'erisation}\\ R\'ealisation du circuit int\'egr\'e MOSTEC \\ Diodes diffus\'ees}
\author{Mohamed Hage Hassan \\ Lucien Dos Santos \\ Nathanaël Marty \\ Ayoub Bargach \\ Benjamin Bony}
\date{14 Mars, 2017}
\maketitle

\tableofcontents
\clearpage

\section{Introduction}


\section{Premi\`ere sc\'eance}

\subsection{Observation zones actives et nettoyage des plaques}

\subsection{Oxydation thermique}

\subsection{Mod\'elisation dopage}

\subsection{\'Etude oxyde de champ}

\subsection{\'Etude dopage : R-carr\'e}

\subsection{Retrait du verre de phosphore}

\subsection{Pr\'esentation Pulv\'erisation cathodique}

\subsection{D\'epot Aluminium : R-carr\'e Alu }

\section{Deuxi\`eme sc\'eance}

\subsection{Etalement r\'esine, photolithographie}

\subsection{Gravure humide de l'aluminium}

\subsection{Nettoyage RIE face arri\`ere}

\subsection{Mesure de motif profilom\`etre}


\section{Conclusion}




\end{document}