\documentclass[11pt]{article}
\renewcommand{\baselinestretch}{1.05}

\usepackage{amsmath,amsthm,verbatim,amssymb,amsfonts,amscd, graphicx}
\usepackage{graphics}

\usepackage{xcolor}

\usepackage[hidelinks]{hyperref}

\usepackage{parskip}

\renewcommand{\contentsname}{Table des mati\`eres}

\topmargin0.0cm
\headheight0.0cm
\headsep0.0cm
\oddsidemargin0.0cm
\textheight23.0cm
\textwidth16.5cm
\footskip1.0cm

\begin{document}

\title{\textbf{Technologie salle blanche et caract\'erisation}\\ R\'ealisation du circuit int\'egr\'e MOSTEC \\ Diodes diffus\'ees}
\author{Mohamed Hage Hassan \\ Lucien Dos Santos \\ Nathana\"el Marty \\ Ayoub Bargach \\ Benjamin Bony}
\date{14 Mars, 2017}
\maketitle

\tableofcontents
\clearpage

\section{Introduction}
Le d\'eveloppement du monde \'electronique se repose sur la miniaturisation des circuits, en accompagnant la loi de Moore. On passe de la micro\'electronique en nano, ce qui a toujours n\'ecessiter des proc\'edures complexes pour la mise en place de tels d\'efis technologiques. 

Ces proc\'edures sont \`a la base de la cr\'eation des salles blanches, qui ont pour but d'\'eviter toute intrusion d'impuret\'es pr\'esentes dans l'air \`a l'int\'erieur des circuits \'electroniques. A cette \'echelle, la moindre particule peut compromettre le fonctionnement d'une puce.

Dans notre cas, on va \'etudier la r\'ealisation des 2 parties principaux d'un transistor MOSFET, l'\'element essentiel d'un circuit \'electronique : la jonction PN, et la capacit\'e MOS. Le pr\'esent compte rendu se focalisera sur la partie diode.


\section{Caract\'erisation mat\'eriaux}

\subsection{Substrat}
\begin{center}
    \noindent \begin{tabular}[!htb]{ | p{7cm} | p{7cm} |}
    \hline
    \textbf{Donn\'ees fabricant} & \textbf{Donn\'ees mesur\'ees} \\
    \begin{itemize}
    	\item[-] Orientation $100 \pm 0.5$ deg
    	\item[-] Dopant
    	\item[-] Epaisseur $e = 275 \pm 25$ µm
    	\item[-] R\'esistivit\'e $\rho = 0.2 - 0.4 \phantom{2} \Omega/cm$
    \end{itemize}
     &
	\begin{itemize}
    	\item[-] R\'esistance de feuille
    		$R_{carre} = 12.4 \phantom{2} \Omega$
    	\item[-] R\'esistivit\'e $\rho = 37.2 \times 10^{-2}  \phantom{2} \Omega/cm$
    	\item[-] Dopage      $N_a = 10^{17} m^{-3}$
    \end{itemize}
    \\
    \hline
    \end{tabular}
\end{center}

\subsection{Oxyde de champs}
	
	\begin{center}
    \noindent \begin{tabular}[!htb]{ | p{7cm} | p{7cm} | }
    \hline
    Epaisseur ellipso $e^{ellipso}$(SiO2) & 527 nm\\ \hline
    Epaisseur profilom\`etre    $e^{profilo}$(SiO2) & 535 nm \\ 
    \hline
    \end{tabular}
    \end{center}
 
\subsection{Diode}

\begin{itemize} \itemsep -2pt
\item[-] R\'esistance de feuille (4 pointes): $R_{carre} = 50.7 \phantom{2} m\Omega$
\item[-] R\'esistance de feuille (induction): $R_{carre} = 8.4 \phantom{2} \Omega$
\end{itemize}

\subsection{Aluminium}

	\begin{center}
    \noindent \begin{tabular}[!htb]{ | p{7cm} | p{7cm} | }
    \hline
    Epaisseur $e_{Al}$ & 449.3 nm \\ \hline
    R\'esistance de feuille $R_{carre}$ & $16.4 \Omega$ \\ \hline
    R\'esistivit\'e       $\rho = R_{carre} \times e_{Al}$ & $7 \times 10^{-6} \Omega/cm$ \\ \hline
    R\'esistivit\'e th\'eorique  $\rho_{th}(Al)$  & $27 \times 10^{-9} \Omega/m$ \\
    \hline
    \end{tabular}
    \end{center}

\section{Fiche Proc\'ed\'es}

\subsection{\'Etude oxyde de champ}

\begin{itemize}
\item \textbf{Technique}\\
La technique utilis\'ee pour cette proc\'edure est l'oxydation thermique humide.

\item \textbf{Calculer les temps de croissance n\'ecessaire pour obtenir 500 nm de SiO2 sur silicium par oxydation humide et oxydation s\`eche \`a 1050 C}\\

L'\'epaisseur d'oxydation varie selon la lois suivante :
\[
    e^{2} + A \times e = B.t
\]

Alors :
\[
t = \frac{e^{2}}{B} + A\frac{e}{B}
\]

avec A et B d\'ependant de la temp\'erature.

D'apr\`es les abaques du cours, \`a 1050 C pour une oxydation humide : 
\begin{itemize} \itemsep -2pt
\item[-] $B = 3 \times 10^{-1} \mu m^{2}/h$
\item[-] $B/A = 2.8 \mu m/h$
\end{itemize}

$\implies t_{humide} = 1 \phantom{2} h$

Pour une oxydation s\`eche :

\begin{itemize} \itemsep -2pt
\item[-] $B = 1.5 \times 10^{-2} \mu m^{2}/h$
\item[-] $B/A = 2 \times 10^{-1} \mu m/h$
\end{itemize}


Alors $t_{seche} = 19 \phantom{2} h$


\item \textbf{Justifier le choix de la technique utilis\'ee}\\
L'oxydation s\`eche permet un \'enorme gain de temps et d'\'energie.
\end{itemize}

\subsection{Nettoyage chimique}

\subsubsection{\'Etapes du proc\'ed\'e} 
Pour supprimer l'oxyde qui se forme naturellement lorsqu'on laisse la plaque \`a l'air libre, on r\'ealise un nettoyage chimique.

\begin{enumerate}\itemsep -2pt
\item La plaque est plong\'ee dans un bain de Hf pendant 5 secondes.
\item Rin\c cage de la plaque \`a l'eau distill\'e puis purifi\'ee
\item Nouvelle oxydation dans un bain d'acide sulfurique + H2O2 pendant 15 mins. 
\item Rin\c cage (cf 2/)
\item Suppression de la couche d'oxyde form\'ee en r\'ealisant \`a nouveau l'\'etape 1/
\item Rin\c cage (cf 2/)
\item S\'echage \`a la centrifugeuse
\end{enumerate}

\subsubsection{R\^ole de chaque \'etape}
On explore aussi la cons\'equence sur le caract\`ere hydrophile ou hydrophobe du silicium :

\begin{enumerate}\itemsep -2pt
\item Supprime l'oxyde naturel. La surface arri\`ere est hydrophile avant corrosion (couche d'oxyde) et hydrophobe apr\`es corrosion (surface de silicium).
\item Stoppe la corrosion de l'oxyde. On observe que la surface rejette l'eau.
\item  Emprisonne les impuret\'es qui auraient pu diffuser dans le silicium dans une nouvelle couche d'oxyde. La couche d'oxyde ainsi form\'e est hydrophile et l'eau reste coll\'e pendant le rin\c cage.
\item Stoppe l'oxydation.
\item Supprime la couche d'oxyde ainsi que les impuret\'es pr\'esentes dedans. Apr\`es cette \'etape la surface arri\`ere est hydrophobe.
\item Arr\^ete la corrosion de l'oxyde. 
\item S\`eche la plaque.
\end{enumerate}


\subsection{Diode -- r\'ealisation de la jonction}

\begin{itemize}

\item \textbf{Technique de dopage utilis\'ee, citer une technique de dopage alternative }\\
Le dopage employ\'e est un dopage par diffusion de $PoCl_3$. \\ Une autre m\'ethode existe: \textit{dopage par implantation ionique}. 

\item \textbf{Profil de temp\'erature, donn\'ees exp\'erimentales}

\end{itemize}

\subsection{D\'ep\^ot des contacts m\'etalliques}

\subsubsection{Description de la technique de d\'ep\^ot} 

La technique employ\'e est la pulv\'erisation cathodique :
\begin{itemize} \itemsep -2pt
\item[-] Tout d'abord les \'echantillons sont plac\'es dans une enceinte sous vide (pression \`a $10^{-7}$ mbar)
\item[-] Un gaz d'argon est lib\'er\'e et ionis\'e dans l'enceinte. 
\item[-] Les ions $Ar^{+}$ sont acc\'el\'er\'es par un champ \'electrique pour venir arracher des atomes d'Al sur une surface plac\'ee au dessus des \'echantillons. 
\item[-] Les atomes d'Al sont acc\'el\'er\'es dans l'autre sens et viennent percuter la surface des \'echantillons pour s'y fixer.
\end{itemize}

\subsubsection{Param\`etres de d\'ep\^ots}

    \noindent \begin{tabular}[!htb]{ | p{4cm} | p{3.5cm} | p{3.5cm} | p{3.5cm} | }
    \hline
    \textbf{Cible} & Aluminium & \textbf{Pression de travail} & $10^{-7}$ mbar\\ \hline
    \textbf{Gaz} & Argon & \textbf{Puissance} &\\ \hline
    \textbf{Couleur du plasma} & Violet & \textbf{Dur\'ee} & 3 min\\
    \hline
    \end{tabular}

\subsubsection{Epaisseur attendue}
    On attend une \'epaisseur d'aluminium de 500 nm environ.

\subsection{Photolithogravure}

\subsubsection{Mesures}

    \noindent \begin{tabular}[!htb]{ | p{7cm} | p{7cm} | }
    \hline
    \textbf{Epaisseur (r\'esine)} & $1.135 \phantom{2}\mu m$ avant recui, $1.052 \phantom{2} \mu m$ apr\`es \\ \hline
    \textbf{Epaisseur (r\'esine grav\'ee + alu)} &  $1.580 \phantom{2} \mu m$\\ \hline
    \textbf{Epaisseur (Alu)} & $0.530 \phantom{2} \mu m$\\
    \hline
    \end{tabular}


\textbf{En d\'eduire la s\'electivit\'e de la gravure :} \\
La gravure attaque l'aluminium mais pas la r\'esine.

\subsubsection{Donn\'ees exp\'erimentales}

    \noindent \begin{tabular}[!htb]{ | p{3.5cm} | p{3.5cm} | p{4cm} | p{3.5cm} | }
    \hline
    \textbf{R\'esine} & S18 13 & \textbf{Dur\'ee (insolation)}  & 7s\\ \hline
    \textbf{Vitesse d'\'etalement} & 5000 rpm & \textbf{Dur\'ee (d\'eveloppement)} & 1 min\\ \hline
    \textbf{T(s\'echage)} & 120 C & \textbf{T(durcissement)} & 130 C\\ \hline
    \textbf{Dur\'ee (s\'echage)} & 2 min 30 s & \textbf{Dur\'ee(durcissement)} & 2 min 30 s \\
    \hline
    \end{tabular}

\begin{itemize}
\item[o] \textbf{Epaisseur de r\'esine attendue} \\
$1.3 \mu m$
\item[o] \textbf{Role du s\'echage}\\
Le s\'echage permet \`a la r\'esine de ne plus \^etre liquide et donc de rester en place sur la plaque, car l'\'epaisseur doit rester constante.
\item[o] \textbf{Role du recuit}\\
Le recuit permet, une fois le d\'eveloppement r\'ealis\'e, de rendre la r\'esine r\'esistante \`a la gravure de l'aluminium.

\end{itemize}

\subsubsection{Gravure aluminium}

\begin{itemize}
\item[o] \textbf{Donn\'ees exp\'erimentales}

	\iffalse
    \noindent \begin{tabular}[!htb]{ | p{7cm} | p{7cm} | }
    \hline
    Bain d'attaque & \\ \hline
    Temp\'erature & \\ \hline
    Dur\'ee & \\
    \hline
    \end{tabular}
    \fi

Gravure \`a l'acide ac\'etique + acide phosphorique \`a 45 C pendant 3 min environ. \`A la fin de la gravure, la face avant de la plaque passe d'une couleur argent\'ee \`a une couleur sombre rapidement.

\item[o] \textbf{En tenant compte de la vitesse de gravure annonc\'ee, estimer l'\'epaisseur d'aluminium} \\ 
        En th\'eorie en 3min on devrait pouvoir retirer $0.9 \mu m$ d'aluminium.

\end{itemize}

\subsubsection{M\'ethode du retrait de r\'esine} 
Gravure de r\'esine \`a l'aide d'un plasma d'$O_2$ qui ne va pas attaquer l'aluminium et le silicium.

\subsection{Nettoyage face arri\`ere}

\subsubsection{Raison de r\'ealisation}
Afin de retirer le dopage \`a l'arri\`ere du wafer. Cette \'etape est importante car elle permet d'\'eviter une diffusion des impuret\'es \`a la surface et un meilleur contact, la face arri\`ere \'etant utilis\'ee comme une masse.

\subsubsection{M\'ethode de mise en place}

On utilise la m\'ethode RIE (Reactive Ion Etching), on combine m\'ecanisme chimique (qui s\'electionne la couche \`a graver) et un m\'ecanisme physique par abrasion anistrope.

Cette m\'ethode fonctionne \`a l'aide d'un r\'eacteur plasma avec deux \'electrodes, le substrat jouant le rôle de la cathode. Le gaz qui entre dans le r\'eacteur est ionis\'e, et sous l'effet de champ, bombarde notre cible.


\clearpage

\section{Fiche Mesures}

\subsection{Ellipsom\'etrie}

\subsubsection{Description de la technique }

L'ellipsom\'etrie permet de mesurer de mani\`ere non-destructive l'\'epaisseur de fines couches de di\'electriques. On envoie une onde monochromatique plane polaris\'ee rectilignement au niveau de la cible, puis on mesure le d\'ephasage ainsi que le rapport des amplitudes entre l'onde incidente et l'onde r\'efl\'echie. 

\subsubsection{Mesure de l'\'epaisseur d'oxyde de champs}

Apr\`es manipulation nous obtenons 5 valeurs d'\'epaisseur :
\begin{itemize} \itemsep -6pt
\item[-] 526.75 nm
\item[-] 521.29 nm
\item[-] 524.87 nm
\item[-] 522.20 nm 
\item[-] 529.75 nm
\end{itemize}
On en d\'eduit une moyenne de 524.97 nm.

\subsection{Profilom\'etrie}

\subsubsection{Description de la m\'ethode}

La profilom\'etrie est une technique pour mesurer la rugosit\'e d'une surface, ou m\^eme sa micro-g\'eom\'etrie. Dans notre cas, une pointe tr\`es fine vient se poser sur un bloc qu'on d\'esire mesurer l'\'epaisseur. Ces blocs sont notamment les plots des diodes qu'on a \'etablie dans les d\'emarches pr\'ec\'edentes. Cette technique est utilis\'ee dans plusieurs \'etapes (apr\`es la gravure humide ainsi que le nettoyage RIE face arri\`ere). 

\subsubsection{Mesure de l'\'epaisseur d'oxyde de champs} 

On mesure l'\'epaisseur de l'oxyde de champs, on retrouve :
\[
	e_{SiO_{2}} = 500 nm
\]

La mesure actuelle est plus pr\'ecise que l'ellipsom\'etrie.
\subsubsection{Mesure de l'\'epaisseur d'aluminium} 

L'\'epaisseur d'aluminium est mesur\'ee \`a :
\[
	e_{SiO_{2}} = 449.3 \phantom{2} nm
\] 

\subsection{Mesure de r\'esistivit\'e -- 4 pointes align\'ees}

\subsubsection{Description de la m\'ethode}
La mesure de r\'esistivit\'e 4 pointes consiste \`a pr\'elever l'intensit\'e sur l'ensemble de l'\'echantillon, et la tension sur une portion de l'\'echantillon. Contrairement \`a une mesure classique de r\'esistance, celle-ci permet de mesurer la r\'esistance r\'eelle de l'\'echantillon (sans consid\'erer les r\'esistances de contact avec les fils). On utilise notamment cette technique pour mesurer des r\'esistivit\'e tr\`es faibles, pour lesquelles les r\'esistances de contacts ne sont pas n\'egligeables par rapport \`a celle de l'\'echantillon.  

Nous n'avons pas utilis\'e la technique de mesure de r\'esistivit\'e 4 pointes, mais la m\'ethode par induction. 
    
\subsection{Mesure de r\'esistivit\'e -- m\'ethode inductive}

\subsubsection {M\'ethodologie }
C'est une technique qui utilise deux bobines. L'une induit des courants de foucault dans notre mat\'eriau, Puis, cela induit un champ magn\'etique, lui m\^eme cr\'eant un courant, dans la bobine plac\'ee \`a proximit\'e. Le courant induit permet alors de r\'ecup\'erer la conductivit\'e sans contact et sans branchement avec le m\'etal.

\subsubsection{Substrat} 

\begin{itemize}
\item[-] $R_{carre}$ du substrat : $ 12.4 \phantom{2} \Omega$
\item[-] Epaisseur de la couche de Si : $0.03 \phantom{2} cm$
\item[-] R\'esistivit\'e du Substrat : $ 0.372 \phantom{2} \Omega.cm$
\end{itemize}


On obtient bien une r\'esistivit\'e compris entre 0.2 et 0.4 Ohm.cm donn\'ees par le fabricant. 

En lisant sur les graphiques (R\'esistivit\'e/Dopage), on obtient un dopage $Na = 10^{17} \phantom{2}  m^{-3}$.

\subsubsection{ Aluminium}

Calcul de la r\'esistance de feuille :

\begin{itemize}
\item[-] $R_{carre}$ total de l'\'echantillon $= R_{feuille}$//$R_{substrat}$
\item[-] $R_{carre}$ total : 8.4 $\Omega$
\end{itemize}

On a alors :
\[
	\frac{1}{R_{totale}} = \frac{1}{R_{feuille}} + \frac{1}{R_{substrat}} 
\]

Ce qui implique que :
\[
	R_{Alu} = 26.04 \Omega
\]

\clearpage

\section{Conclusion}

Au cours de la s\'eance et afin de r\'ealiser une jonction PN munie d'un point de contact m\'etallique, nous avons d\^u suivre toute une proc\'edure. Chaque \'etape n\'ecessitait d'\^etre pr\'ecis et concentr\'e (pour ne pas inverser deux sous \'etapes ou m\^eme casser une plaque). Au final cette s\'eance montre bien les exigences qu'imposent les proc\'ed\'es d'int\'egration en salle blanche: Chaque \'etape est suivie d'une v\'erification par la mesure et la propret\'e de l'environnement de travail se doit d'\^etre irr\'eprochable. De plus, certaines \'etapes n'appartiennent pas au proc\'ed\'es de fabrication \`a proprement parl\'e mais sont quand m\^eme indispensable, comme le nettoyage des plaques par exemple.

Enfin pour v\'erifier que la jonction PN final r\'epond bien aux attentes, nous proc\'edons \`a une caract\'erisation lors d'une autre s\'eance hors salle blanche.



\end{document}