\documentclass[11pt]{article}
\renewcommand{\baselinestretch}{1.05}

\usepackage{amsmath,amsthm,verbatim,amssymb,amsfonts,amscd, graphicx}
\usepackage{graphics}

\usepackage{xcolor}

\usepackage[hidelinks]{hyperref}

\usepackage{parskip}

\renewcommand{\contentsname}{Table des mati\`eres}

\topmargin0.0cm
\headheight0.0cm
\headsep0.0cm
\oddsidemargin0.0cm
\textheight23.0cm
\textwidth16.5cm
\footskip1.0cm

\begin{document}

\title{\textbf{Technologie salle blanche et caract\'erisation}\\ R\'ealisation du circuit int\'egr\'e MOSTEC \\ Diodes diffus\'ees}
\author{Mohamed Hage Hassan \\ Lucien Dos Santos \\ Nathanaël Marty \\ Ayoub Bargach \\ Benjamin Bony}
\date{14 Mars, 2017}
\maketitle

\tableofcontents
\clearpage

\section{Introduction}
Le d\'eveloppement du monde \'electronique se repose sur la miniaturisation des circuits, en accompagnant la loi de Moore. On passe de la micro\'electronique en nano, ce qui a toujours n\'ecessiter des proc\'edures complexes pour la mise en place de tels d\'efis technologiques. 

Ces proc\'edures sont \`a la base de la creation des salles blanches, qui ont pour aussi pour but d'\'eviter l'introduction des impurt\'ees provenant de l'atmosph\`ere dans les circuits \'electroniques.

Dans notre cas, on va \'etudier la r\'ealisation des 2 parties principaux d'un transistor MOSFET, l'\'element essentiel d'un circuit \'electronique : la jonction PN, et la capacit\'e MOS. Le pr\'esent compte rendu se focalisera sur la partie diode.


\section{Caract\'erisation mat\'eriaux}

\subsection{Substrat}
\begin{center}
    \noindent \begin{tabular}[!htb]{ | p{7cm} | p{7cm} |}
    \hline
    \textbf{Donn\'ees fabricant} & \textbf{Donn\'ees mesur\'ees} \\
    \begin{itemize}
    	\item[-] Orientation $100 \pm 0.5$ deg
    	\item[-] Dopant
    	\item[-] Epaisseur $e = 275 \pm 25$ µm
    	\item[-] R\'esistivit\'e $\rho = 0.2 - 0.4 \phantom{2} \Omega/cm$
    \end{itemize}
     &
	\begin{itemize}
    	\item[-] R\'esistance de feuille
    		$R_{carre} = 12.4 \phantom{2} \Omega$
    	\item[-] R\'esistivit\'e $\rho = 37.2 \times 10^{-2}  \phantom{2} \Omega/cm$
    	\item[-] Dopage      $N_a = 10^{17} m^{-3}$
    \end{itemize}
    \\
    \hline
    \end{tabular}
\end{center}

\subsection{Oxyde de champs}
	
	\begin{center}
    \noindent \begin{tabular}[!htb]{ | p{7cm} | p{7cm} | }
    \hline
    Epaisseur ellipso $e^{ellipso}$(SiO2) & \\ \hline
    Epaisseur profilom\`etre    $e^{profilo}$(SiO2) & \\ 
    \hline
    \end{tabular}
    \end{center}

\subsection{Capacit\'e}

Epaisseur oxyde: $e_{oxyde}$ = 

\subsection{Diode}

Dopant
\begin{itemize} \itemsep -2pt
\item[-] R\'esistance de feuille (4 pointes): $R_{carre} = 50.7 \phantom{2} m\Omega$
\item[-] R\'esistance de feuille (induction): $R_{carre} = 8.4 \phantom{2} \Omega$
\end{itemize}

\subsection{Aluminium}

	\begin{center}
    \noindent \begin{tabular}[!htb]{ | p{7cm} | p{7cm} | }
    \hline
    Epaisseur $e_{Al}$ & 449.3 nm \\ \hline
    R\'esistance de feuille $R_{carre}$ & $16.4 \Omega$ \\ \hline
    R\'esistivit\'e       $\rho = R_{carre} \times e_{Al}$ & $7 \times 10^{-6} \Omega/cm$ \\ \hline
    R\'esistivit\'e th\'eorique  $\rho_{th}(Al)$  & $27 \times 10^{-9} \Omega/m$ \\
    \hline
    \end{tabular}
    \end{center}

\section{Fiche Proc\'ed\'es}

\subsection{\'Etude oxyde de champ}

\begin{itemize}
\item \textbf{Technique}\\
La technique utilis\'ee pour cette proc\'edure est l'oxydation thermique humide.

\item \textbf{Calculer les temps de croissance n\'ecessaire pour obtenir 500 nm de SiO2 sur silicium par oxydation humide et oxydation s\`eche \`a 1050 C}\\

L'\'epaisseur d'oxydation varie selon la lois suivante :
\[
    e^{2} + A \times e = B.t
\]

Alors :
\[
t = \frac{e^{2}}{B} + A\frac{e}{B}
\]

avec A et B d\'ependant de la temp\'erature.

D'apr\`es les abaques du cours, \`a 1050 C pour une oxydation humide : 
\begin{itemize} \itemsep -2pt
\item[-] $B = 3 \times 10^{-1} \mu m^{2}/h$
\item[-] $B/A = 2.8 \mu m/h$
\end{itemize}

$\implies t_{humide} = 1 \phantom{2} h$

Pour une oxydation s\`eche :

\begin{itemize} \itemsep -2pt
\item[-] $B = 1.5 \times 10^{-2} \mu m^{2}/h$
\item[-] $B/A = 2 \times 10^{-1} \mu m/h$
\end{itemize}


Alors $t_{seche} = 19 \phantom{2} h$


\item \textbf{Justifier le choix de la technique utilis\'ee}\\
L'oxydation s\`eche permet un \'enorme gain de temps et d'\'energie.
\end{itemize}

\subsection{Nettoyage chimique}

\subsubsection{\'Etapes du proc\'ed\'e} 
Pour supprimer l'oxyde qui se forme naturellement lorsqu'on laisse la plaque \`a l'air libre, on r\'ealise un nettoyage chimique.

\begin{enumerate}\itemsep -2pt
\item La plaque est plong\'ee dans un bain de Hf pendant 5 secondes.
\item Rin\c cage de la plaque \`a l'eau distill\'e puis purifi\'ee
\item Nouvelle oxydation dans un bain d'acide sulfurique + H2O2 pendant 15 mins. 
\item Rin\c cage (cf 2/)
\item Suppression de la couche d'oxyde form\'ee en r\'ealisant \`a nouveau l'\'etape 1/
\item Rin\c cage (cf 2/)
\item S\'echage \`a la centrifugeuse
\end{enumerate}

\subsubsection{R\^ole de chaque \'etape}
On explore aussi la cons\'equence sur le caract\`ere hydrophile ou hydrophobe du silicium :

\begin{enumerate}\itemsep -2pt
\item Supprime l'oxyde naturel. La surface arri\`ere est hydrophile avant corrosion (couche d'oxyde) et hydrophobe apr\`es corrosion (surface de silicium).
\item Stoppe la corrosion de l'oxyde. On observe que la surface rejette l'eau.
\item  Emprisonne les impuret\'es qui auraient pu diffuser dans le silicium dans une nouvelle couche d'oxyde. La couche d'oxyde ainsi form\'e est hydrophile et l'eau reste coll\'e pendant le rin\c cage.
\item Stoppe l'oxydation.
\item Supprime la couche d'oxyde ainsi que les impuret\'es pr\'esentes dedans. Apr\`es cette \'etape la surface arri\`ere est hydrophobe.
\item Arr\^ete la corrosion de l'oxyde. 
\item S\`eche la plaque.
\end{enumerate}


\subsection{Diode -- r\'ealisation de la jonction}

\begin{itemize}

\item \textbf{Technique de dopage utilis\'ee, citer une technique de dopage alternative }\\
Le dopage employ\'e est un dopage par diffusion de $PoCl_3$. \\ Une autre m\'ethode existe: \textit{dopage par implantation ionique}. 

\item \textbf{Profil de temp\'erature, donn\'ees exp\'erimentales}

\end{itemize}

\subsection{D\'ep\^ot des contacts m\'etalliques}

\subsubsection{Description de la technique de d\'ep\^ot} 

La technique employ\'e est la pulv\'erisation cathodique :
\begin{itemize} \itemsep -2pt
\item[-] Tout d'abord les \'echantillons sont plac\'es dans une enceinte sous vide (pression \`a $10^{-7}$ mbar)
\item[-] Un gaz d'argon est lib\'er\'e et ionis\'e dans l'enceinte. 
\item[-] Les ions $Ar^{+}$ sont acc\'el\'er\'es par un champ \'electrique pour venir arracher des atomes d'Al sur une surface plac\'ee au dessus des \'echantillons. 
\item[-] Les atomes d'Al sont acc\'el\'er\'es dans l'autre sens et viennent percuter la surface des \'echantillons pour s'y fixer.
\end{itemize}

\subsubsection{Param\`etres de d\'ep\^ots}

    \noindent \begin{tabular}[!htb]{ | p{4cm} | p{3.5cm} | p{3.5cm} | p{3.5cm} | }
    \hline
    \textbf{Cible} & Aluminium & \textbf{Pression de travail} & $10^{-7}$ mbar\\ \hline
    \textbf{Gaz} & Argon & \textbf{Puissance} &\\ \hline
    \textbf{Couleur du plasma} & Violet & \textbf{Dur\'ee} & 3 min\\
    \hline
    \end{tabular}

\subsubsection{Epaisseur attendue}
    On attend une \'epaisseur d'aluminium de 500 nm environ.

\subsection{Photolithogravure}

\subsubsection{Mesures}

    \noindent \begin{tabular}[!htb]{ | p{7cm} | p{7cm} | }
    \hline
    \textbf{Epaisseur (r\'esine)} & $1.135 \phantom{2}\mu m$ avant recui, $1.052 \phantom{2} \mu m$ apr\`es \\ \hline
    \textbf{Epaisseur (r\'esine grav\'ee + alu)} &  $1.580 \phantom{2} \mu m$\\ \hline
    \textbf{Epaisseur (Alu)} & $0.530 \phantom{2} \mu m$\\
    \hline
    \end{tabular}


\textbf{En d\'eduire la s\'electivit\'e de la gravure :} \\
La gravure attaque l'aluminium mais pas la r\'esine.

\subsubsection{Donn\'ees exp\'erimentales}

    \noindent \begin{tabular}[!htb]{ | p{3.5cm} | p{3.5cm} | p{4cm} | p{3.5cm} | }
    \hline
    \textbf{R\'esine} & S18 13 & \textbf{Dur\'ee (insolation)}  & 7s\\ \hline
    \textbf{Vitesse d'\'etalement} & 5000 rpm & \textbf{Dur\'ee (d\'eveloppement)} & 1 min\\ \hline
    \textbf{T(s\'echage)} & 120 C & \textbf{T(durcissement)} & 130 C\\ \hline
    \textbf{Dur\'ee (s\'echage)} & 2 min 30 s & \textbf{Dur\'ee(durcissement)} & 2 min 30 s \\
    \hline
    \end{tabular}

\begin{itemize}
\item[o] \textbf{Epaisseur de r\'esine attendue} \\
$1.3 \mu m$
\item[o] \textbf{Role du s\'echage}\\
Le s\'echage permet \`a la r\'esine de ne plus \^etre liquide et donc de rester en place sur la plaque, car l'\'epaisseur doit rester constante.
\item[o] \textbf{Role du recuit}\\
Le recuit permet, une fois le d\'eveloppement r\'ealis\'e, de rendre la r\'esine r\'esistante \`a la gravure de l'aluminium.

\end{itemize}

\subsubsection{Gravure aluminium}

\begin{itemize}
\item[o] \textbf{Donn\'ees exp\'erimentales}

	\iffalse
    \noindent \begin{tabular}[!htb]{ | p{7cm} | p{7cm} | }
    \hline
    Bain d'attaque & \\ \hline
    Temp\'erature & \\ \hline
    Dur\'ee & \\
    \hline
    \end{tabular}
    \fi

Gravure \`a l'acide ac\'etique + acide phosphorique \`a 45 C pendant 3 min environ. \`A la fin de la gravure, la face avant de la plaque passe d'une couleur argent\'ee \`a une couleur sombre rapidement.

\item[o] \textbf{En tenant compte de la vitesse de gravure annonc\'ee, estimer l'\'epaisseur d'aluminium} \\ 
        En th\'eorie en 3min on devrait pouvoir retirer $0.9 \mu m$ d'aluminium.

\end{itemize}

\subsubsection{M\'ethode du retrait de r\'esine} 
Gravure de r\'esine \`a l'aide d'un plasma d'$O_2$ qui ne va pas attaquer l'aluminium et le silicium.

\subsection{Nettoyage face arri\`ere}

\subsubsection{Raison de r\'ealisation}
Afin de retirer le dopage \`a l'arri\`ere du wafer. Cette \'etape est importante par elle permet d'\'eviter une diffusion des impuret\'es \`a la surface et un meilleur contact, la face arri\`ere \'etant utilis\'ee comme une masse.

\subsubsection{M\'ethode de mise en place}

On utilise la m\'ethode RIE (Reactive Ion Etching), on combine m\'ecanisme chimique (qui s\'electionne la couche \`a graver) et m\'ecanisme physique par abrasion anistrope.

Cette m\'ethode fonctionne \`a l'aide d'un r\'eacteur plasma avec deux \'electrodes, le substrat jouant le rôle de la cathode. Le gaz qui entre dans le r\'eacteur est ionis\'e, et sous l'effet de champ, bombarde notre cible.


\clearpage

\section{Fiche Mesures}

\subsection{Ellipsom\'etrie}

\subsubsection{ Description de la technique }

L'ellipsom\'etrie permet de mesurer de manière non-destructive l'\'epaisseur de fines couches de di\'electriques. On envoie une onde monochromatique plane polaris\'ee rectilignement au niveau de la cible, puis on mesure le d\'ephasage ainsi que le rapport des amplitudes entre l'onde incidente et l'onde r\'efl\'echie. 

\subsubsection{Mesure de l'\'epaisseur d'oxyde de champs}
Au moins 5 points de mesure diff\'erents, calcul de moyenne et d\'eviation standard
\subsubsection{Mesure de l'\'epaisseur d'oxyde mince (capacit\'e)}


\subsection{Profilom\'etrie}

\begin{itemize}
\item \textbf{ D\'ecrire la technique  }
\item \textbf{ Mesure de l'\'epaisseur d'oxyde de champs \\ 
 Comparer \`a la mesure pr\'ec\'edente}
\item \textbf{ Mesure de l'\'epaisseur d'aluminium. En d\'eduire la vitesse de d\'ep\^ot.}
\end{itemize}

\subsection{Mesure de r\'esistivit\'e -- 4 pointes align\'ees}

\begin{itemize}
\item \textbf{ D\'ecrire la technique  }
\item \textbf{ Substrat \\ 
  Tableau de mesure, graphique, r\'esistance de feuille, r\'esistivit\'e, dopage}
\item \textbf{ Jonction PN (t\'emoin diffus\'e pleine plaque ou t\'emoin implant\'e pleine plaque) \\ 
Tableau de mesure, graphique, r\'esistance de feuille}
\end{itemize}

    

\subsection{Mesure de r\'esistivit\'e – m\'ethode inductive}

\begin{itemize}
\item \textbf{ D\'ecrire la technique  }
\item \textbf{ Substrat \\ 
   r\'esistance de feuille, comparer aux r\'esultats pr\'ec\'edemment obtenus}
\item \textbf{ Aluminium \\ 
r\'esistance de feuille, r\'esistivit\'e ; justifier que la r\'esistance de feuille mesur\'ee est bien celle de
     l'aluminium}
\item \textbf{Jonction PN (t\'emoin diffus\'e ou implant\'e pleine plaque) \\
    r\'esistance de feuille.}
\end{itemize}


\iffalse
\subsection{Capacit\'e -- r\'ealisation du di\'electrique}

\begin{itemize}

\item \textbf{Technique utilis\'ee, justifier}


\item \textbf{Profil de temp\'erature, expliquer le r\^ole des \'etapes-cl\'es}

\item \textbf{Calcul de l'\'epaisseur th\'eorique d'oxyde, comparer \`a la valeur mesur\'ee. Quelle est l'\'epaisseur de Si
        consomm\'e pour former cette couche de SiO2 ?}
\end{itemize}
\fi 


\iffalse

\subsection{Observation zones actives et nettoyage des plaques}

\subsection{Oxydation thermique}

\subsection{Mod\'elisation dopage}

\subsection{\'Etude dopage : R-carr\'e}

\subsection{Retrait du verre de phosphore}

\subsection{Pr\'esentation Pulv\'erisation cathodique}

\subsection{D\'epot Aluminium : R-carr\'e Alu }

\section{Deuxi\`eme sc\'eance}

\subsection{Etalement r\'esine, photolithographie}

\subsection{Gravure humide de l'aluminium}

\subsection{Nettoyage RIE face arri\`ere}

\subsection{Mesure de motif profilom\`etre}
\fi

\clearpage

\section{Conclusion}




\end{document}